% Этот шаблон документа разработан в 2014 году 

\documentclass[a4paper,14pt]{article}

%%% Работа с русским языком
\usepackage{cmap}					% поиск в PDF
\usepackage{mathtext} 				% русские буквы в фомулах
\usepackage[T2A]{fontenc}			% кодировка
\usepackage[utf8]{inputenc}			% кодировка исходного текста
\usepackage[english,russian]{babel}	% локализация и переносы

%%% Дополнительная работа с математикой
\usepackage{amsmath,amsfonts,amssymb,amsthm,mathtools} % AMS
\usepackage{icomma} % "Умная" запятая: $0,2$ --- число, $0, 2$ --- перечисление

%% Номера формул
%\mathtoolsset{showonlyrefs=true} % Показывать номера только у тех формул, на которые есть \eqref{} в тексте.
%\usepackage{leqno} % Немуреация формул слева

%% Свои команды
\DeclareMathOperator{\sgn}{\mathop{sgn}}

%% Перенос знаков в формулах (по Львовскому)
\newcommand*{\hm}[1]{#1\nobreak\discretionary{}
{\hbox{$\mathsurround=0pt #1$}}{}}

%%% Работа с картинками
\usepackage{graphicx}  % Для вставки рисунков
\graphicspath{}  % папки с картинками
\setlength\fboxsep{3pt} % Отступ рамки \fbox{} от рисунка
\setlength\fboxrule{1pt} % Толщина линий рамки \fbox{}
\usepackage{wrapfig} % Обтекание рисунков текстом

%%% Работа с таблицами
\usepackage{array,tabularx,tabulary,booktabs} % Дополнительная работа с таблицами
\usepackage{longtable}  % Длинные таблицы
\usepackage{multirow} % Слияние строк в таблице

%%% Теоремы
\theoremstyle{plain} % Это стиль по умолчанию, его можно не переопределять.
\newtheorem{theorem}{Теорема}[section]
\newtheorem{proposition}[theorem]{Утверждение}
 
\theoremstyle{definition} % "Определение"
\newtheorem{corollary}{Следствие}[theorem]
\newtheorem{problem}{Задача}[section]
 
\theoremstyle{remark} % "Примечание"
\newtheorem*{nonum}{Решение}

%%% Программирование
\usepackage{etoolbox} % логические операторы

%%% Страница
\usepackage{extsizes} % Возможность сделать 14-й шрифт
\usepackage{geometry} % Простой способ задавать поля
	\geometry{top=20mm}
	\geometry{bottom=20mm}
	\geometry{left=30mm}
	\geometry{right=15mm}
 %
\usepackage{fancyhdr} % Колонтитулы
 	\pagestyle{fancy}
 	\renewcommand{\headrulewidth}{1pt}  % Толщина линейки, отчеркивающей верхний колонтитул
 	%\lfoot{Нижний левый}
 	%\rfoot{Нижний правый}
 	\rhead{\arabic{page}}
 	%\chead{Верхний в центре}
 	%\lhead{\includegraphics[width=15mm]{MADness}}
 	\cfoot{} % По умолчанию здесь номер страницы

\usepackage{setspace} % Интерлиньяж
%\onehalfspacing % Интерлиньяж 1.5
%\doublespacing % Интерлиньяж 2
%\singlespacing % Интерлиньяж 1

\usepackage{lastpage} % Узнать, сколько всего страниц в документе.

\usepackage{soulutf8} % Модификаторы начертания

\usepackage{hyperref}
\usepackage[usenames,dvipsnames,svgnames,table,rgb]{xcolor}
\hypersetup{				% Гиперссылки
    unicode=true,           % русские буквы в раздела PDF
    pdftitle={Заголовок},   % Заголовок
    pdfauthor={Автор},      % Автор
    pdfsubject={Тема},      % Тема
    pdfcreator={Создатель}, % Создатель
    pdfproducer={Производитель}, % Производитель
    pdfkeywords={keyword1} {key2} {key3}, % Ключевые слова
    colorlinks=true,       	% false: ссылки в рамках; true: цветные ссылки
    linkcolor=red,          % внутренние ссылки
    citecolor=green,        % на библиографию
    filecolor=magenta,      % на файлы
    urlcolor=cyan           % на URL
}

%\renewcommand{\familydefault}{\sfdefault} % Начертание шрифта

\usepackage{multicol} % Несколько колонок

\author{dem[]n}
\title{Шаблон документа}
\date{\today}

\begin{document} % конец преамбулы, начало документа

\thispagestyle{empty} %Обеспечивает отсутствие колонтитулов на этой странице

%\begin{tabulary}{\textwidth}{J|C}
%\huge{\textbf{БИБЛИОТЕЧКА dem[]n'a}} & %\includegraphics[width=0.6\linewidth]{MADness} \\
%\end{tabulary}

\begin{center}

\textbf{ИНСТИТУТ НИЗКИХ И ПРИЗЕМЛЕННЫХ ИССЛЕДОВАНИЙ РОССИЙСКОЙ АКАДЕМИИ ЛЖЕНАУК}

\begin{flushright}
На правах рукописи
\end{flushright}

\vspace{10ex}

\textbf{Темнов Марк Злодеевич}

\vspace{10ex}

\textbf{{\large ИССЛЕДОВАНИЕ ПРОЦЕССОВ ФИЛЬТРАЦИИ СМЕСЕЙ УГЛЕВОДОРОДОВ В ПОРИСТОЙ СРЕДЕ С УЧЕТОМ ФАЗОВЫХ ПЕРЕХОДОВ}}

\vspace{10ex}

Специальность 01.02.05 --- Механика жидкости, газа и плазмы

\vspace{10ex}

Диссертация на соискание ученой степени кандидата физико-математических наук

\vspace{10ex}

\begin{flushright}
Научный руководитель: 

доктор технических наук, 

Кроликов Е.Б.
\end{flushright}

\vfill

Москва --- 2017

\end{center}

\newpage

\section{Экспериментальные исследования}

Физическое моделирование описанных процессов проходило на экспериментальной установке <<Пласт>>. Данная установка позволяет моделировать одномерный процесс фильтрации при натурных термобарических условиях: температура --- до $200~^{\circ}C$ , давление --- до $28$~МПа. Схема данной установке указана на рисунке~1.

\subsection{Методика эксперимента}
 
Для начала требуется приготовить модельную смесь такого состава, чтобы при фильтрации через модельный участок смесь проходила через ретроградный участок. Смесь готовится непосредственно в цилиндрах высокого давления. Сначала туда заливается пентан, затем закачивается метан, затем происходит увеличение давления до закритических значений для смеси данного состава при данной температуре. После этого для равномерного распределения смеси между двумя цилиндрами, а так же для перемешивания смеси происходит периодический сброс давления и поднятие его до закритических значений. Процесс перемешивания продолжается полчаса. Затем открывается кран на входе в экспериментальный участок, и смесь с закритическими параметрами заполняет поровое пространство экспериментального участка. Кран на выходе из экспериментального участка держится закрытым, чтобы смесь не перешла в двухфазное состояние раньше времени. После выравнивания давления по длине участка начинаем медленно открывать регуляторы давления на входе в экспериментальный участок и на выходе из него, тем самым добиваясь нужного расхода и перепада давления на участке. После этого начинаем запись данных по давлению и расходу, т.е. регистрируем экспериментальные данные. 

С расчетом состава смеси есть некоторые трудности. По совету старших коллег я считал процесс расширения метана из баллона в цилиндр изотермическим. Но вот совсем недавно мне пришло в голову, что он таковым не является. Вообще, процесс политропический, но показатель политропы хрен рассчитаешь, поэтому возьмем хотя бы за основу адиабатическое расширение. Сталь все же не слишком теплопроводна, так что будет считать, что тепловых потерь на нагрев окружающей среды нет. Но смена процесса несет за собой существенные изменения в расчете. Дело в том, что я рассматриваю неидеальный газ, уравнение состояния которого имеет вид $pV = z(p,T)\frac{m}{\mu}RT$. В случае изотермического процесса мы имеем одно уравнение с одним неизвестным --- p, которое можно решить итеративным способом, на каждом шаге рассчитывая давление и коэффициент сжимаемости z. В случае адиабатного процесса будет меняться как температура газа, так и его давление, поэтому у нас будет одно уравнение с двумя неизвестными, которое неопределено. 

Дабы разрешить эту проблему, сначала нужно оценить, насколько она серьезна. Поэтому я собираюсь сначала рассчитать изотермический процесс в предположении идеального газа, затем рассчитать адиабатический процесс опять же для идеального газа. Это поможет оценить изменение температуры в таком процессе. Думаю, для моих диапазонов изменения температуры (а это даже не порядок) зависимостью коэффициента адиабаты от температуры (и давления, наверное, я уже не помню) можно пренебречь. 

\subsubsection{Оценка процессов при перемешивании смеси}

Рассмотрим части установки, которые задействованы в приготовлении модельной смеси. Это баллон с метаном и цилиндры высокого давления. Баллон с метаном является стандартным изделием, его объем составляет 40 л ($V_t = 40$~л = $4 \cdot 10^{-2}$~м$^3$). Два цилиндра имеют общий объем 10 л ($V_c = 10$~л = $10^{-2}$~м$^3$). Давление в баллоне примем равным 150 бар ($p_t = 150$~бар = $15 \cdot 10^6$~Па). Температура окружающей среды составляет 20$^\circ$C ($T_1 = 293$~К). Универсальная газовая постоянная $R = 8,31 \frac{\textrm{Дж}}{\textrm{моль}\cdot\textrm{K}}$ 

\subsubsection{Изотермический процесс}

Сначала рассмотрим случай изотермического расширения идеального газа. Для такого газа справедливо уравнение состояния Клайперона-Менделеева:

\begin{equation}
	p_1 V_t = \nu_1 R T_1,
\end{equation}
где $\nu_1$ --- количество вещества метана в баллоне. По определению $\nu = \frac{m}{\mu}$, где $m$ --- масса метана, $\mu = 16 \cdot 10^{-3}$~$\frac{\textrm{кг}}{\textrm{моль}}$ --- молярная масса метана.

Определим количество вещества:

\begin{equation}
	\nu_1 = \frac{p_1 V_t}{R T_1} = \frac{15 \cdot 10^6 \cdot 4 \cdot 10^{-2}}{8,31 \cdot 293} = 246,42~\textrm{моль}.
\end{equation} 

Определим массу метана:

\begin{equation}
	m_1 = \mu \cdot \nu_1 = 16 \cdot 10^{-3} \cdot 246,42 = 3,94~\textrm{кг}.
\end{equation}

Определим давление, которое установится в системе после открытия крана между баллоном и цилиндром:

\begin{equation}
	p_2 = \frac{\nu_1 \cdot R \cdot T_1}{V_t + V_c} = \frac{246,42 \cdot 8,31 \cdot 293}{4 \cdot 10^{-2} + 10^{-2}} = 12 \cdot 10^6~\textrm{Па}.
\end{equation}

\subsubsection{Адиабатический процесс}

Будем считать, что коэффициент адиабаты $k$ не зависит от давления и температуры, т.е $k = const$. 
%\begin{wrapfigure}{c}{0.1\linewidth}
%	\includegraphics[width=\linewidth]{MADness}
%\end{wrapfigure}

\newpage

13
14

\end{document} % конец документа

